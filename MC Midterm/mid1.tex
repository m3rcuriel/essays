\documentclass{article}

\usepackage[super]{nth}
\usepackage{ifpdf}
\usepackage{mla}

\begin{document}
\begin{mla}{Lee}{Mracek}{Mary Hill}{American Studies}{\today}{Tempered in the Marne}
    When our founders signed the Constitution of the United States of America, some of them, at least, had a goal of an isolated nation that did not interfere in the sovereignty of its neighbors. John Monroe, the sixth president, signed this idea into law with the Monroe Doctrine, and set a course for America's foreign relations that would not change course until the early 1900s. Public opinion, however, and certainly legal practice, began to change in the tumultuous years between 1914 to 1917, and again during the period of the Second World War from 1939 to 1945, due to shifting immigration pressures and a change in both American global status and in the entitlement of American leaders.
    
    Before the First World War, America had arguably only dropped the policy of isolationism once: during the Spanish-American War of 1898. Before that war, journalists had been working for years to push public opinion away from sympathy for the Spanish occupation of Cuba and began to practice yellow journalism in an attempt to sway American opinion towards intervention, and perhaps towards an American occupation of Cuba in an attempt to control Cuba's massive agricultural resources. After perhaps twenty years of journalistic work, corporate and lobbying from the vocal Cuban immigrants forced -- or bought -- President McKinley's hand and he declared war. The war ended after three months of American domination, and the American jingoists were able to ride the swell of public opinion into establishing the idea of America as a force of global justice, and discarding the antiquated ideas of isolationism. Unfortunately, the next few years would end any dream of massive American imperialism abroad, and allow the pacifists to curtail their more aggressive compatriots. This disruption came in the form of the Philippine-American War.

    One of the strongest Spanish military bases when war broke out dwelt in the Philippines, a mostly autonomous uncolonized nation technically under the control of the Spanish. The U.S Navy won a decisive victory, and essentially found themselves the new overlords of the Philippines. In similar fashion to the occupation of Cuba, American forces moved in, and quickly realized that the rest of the world did not quite agree with the Western ideas of Social Darwinism. In fact, the Filipinos had up to this point believed that the U.S Government would grant them either independence or autonomy and certainly not try to subjugate them. As a result, the Philippine-American War turned into a drawn our guerilla war. The combination of a drawn out brutal war and modernized media created a rapid reversal in public opinion that forced America back into its shell of isolationism.

    The start of World War I has often been called ``the shot heard round the world'' because in a way it did open with a bang. Within the span of a few months, all the major powers of the world were embroiled in a massive land, sea, and aerial war. The U.S, at least at the beginning, wanted to stay out. The American majority felt as though they had no stake in a war fought across the Atlantic Ocean, and did not want to waste money and lives fighting it. Additionally, they had no idea which side to enter on because of the extreme cultural diffusion in the United States. The aristocratic elite supported the Allies, of course, but native Germans, Austrians, and Turks had been immigrating to the United States since it first opened the borders and each maintained a fragment of nationalistic fervor for their own country. President Wilson, re-elected in 1916, promised to continue his previous pacifistic policies based on American popular opinion. However, the general American leadership strongly opposed him and believed that entering the war was both moral and necessary especially considering America's position as a new world power. Teddy Roosevelt, especially, believed that America was morally responsible for the fate of the world and was required to intervene. Some felt as though America had to justify or use its power in order to vindicate itself as a world power.

    Public opinion in America at the start of World War II, however, contrasts strongly with the split in American opinion before World War I. Immediately after World War I, America was launched into a period known as the Jazz Age: a huge outpouring of wealth and culture as exposed in Fitzgerald's writing. In his novel, \textit{The Great Gatsby}, he portrays the culture of the time as one of isolated carelessness both through the characters of Jordan Baker and Daisy Buchanan, as well as through his description of Tom and Daisy's casual aloofness in the face of fractured reality. Fitzgerald describes Tom and Daisy as carelessly rich, which enabled them to go, break things, and run off again in search of more things to play with. This was a time of quick wealth and a large rich upper class which lorded over the poor, which Fitzgerald captured with the childish natures of his characters. Once the NY Stock Market crashed in 1929, the carelessly wealthy aristocrats were no longer able to survive. Tom could not simply ``become a bondsman'' because there was nothing to invest in. The money represented in West Egg just dried up and cast wide swaths of America into a struggle to survive. Poor and rural families such as the Joads did not want to have a war because to them, it makes no sense to embroil America in foreign affairs until domestic issues were resolved, especially because a war means shipping off the breadwinner of the family to fight for years and possibly die. Fortunately, by this time, Franklin Roosevelt had begun to force America out of the Great Depression, but feared being cast back into it if America entered the war. As Steinbeck portrays though his narrative of the migrant worker and his turtle analogy, America as a world power and as a culture does not yet understand where it was going, but definitely did not want to get thrown into a war while facing this uncertainty. Much like the turtle in \textit{The Grapes of Wrath}, America moved purposefully, even though uncertainty remained as to its final destination. The police brutality and cruelty experienced by the Joads also warned people that during a war, Congress would find it easy to pass bills which suppressed and limited American freedoms such as the Espionage and Sedition Acts which helped to enable police brutality.
    
    Before World War I, half of America wanted to go to war. Before World War II, very few Americans wanted to go to war. In both cases, they felt as though domestic issues should come first, and after the World War I, in the Great Depression, enough Americans were feeling those domestic problems that war no longer seemed to be the grand idea that it was to the ex-middle class. After the horrific coverage of the Philippine-American War, and the horrors of modern warfare, it simply no longer seemed worth it. The American perception of the war had changed, and only changed again through the tragedy of December \nth{7}, 1942.
\end{mla}
\begin{mla}{Lee}{Mracek}{Mary Hill}{American Studies}{\today}{title}
        After Victory in Europe Day, most European nations withdrew from the fighting in World War II to concentrate on recovery after the devastation left as a result of the blitzkrieg and years of bombing. On the Pacific front, however, the United States and the USSR continued to fight violent battle after violent battle in a massive campaign towards the mainland. After the United States occupied Iwo Jima in early 1945, American air support kept Tokyo and other major Japanese cities under constant threat of massive firebomb attacks throughout the entire remainder of the war. In the summer of 1945, all of this would change when the United States developed the atomic bomb and bombed two cities in Japan to force an early surrender.

              In the United States, the options for ending the war in the Pacific seemed limited. Either the US forces would have to island hop all the way to Tokyo, while taking massive casualties and possibly extinguishing millions of Japanese lives, although the leadership did not consider that figure while making a decision. In 1945, however, another option was made available my research done at the top secret Manhattan project: the atomic bomb. The atomic bomb was the single largest destructive device ever developed and seemed like a quick and easy way to end the war in Japan. For Truman, as well as the others who were aware of the project, the decision to drop it was not an easy one. On one hand, Japan had offered no indication of surrender, and the bomb seemed like a simple way to convince the Hirohito and Tojo that the war could no longer be won, or even survived, by Japan. The Japanese warplan at the time was essentially to hold off the US until they could get a favorable surrender, a situation which the presence of the bomb removed. The other option for President Truman was to ask the Soviets to enter into the war in Japan from Hokkaido to draw Japanese reinforcements north rather than into the islands. America, however, wanted to avoid this because it meant possibly having the split the occupation of Japan like they were forced to do in Germany.

                 One thing many modern and contemporary sources fail to consider with regards to Truman's decision to drop the bomb is the immense casualties caused by the frequent firebomb attacks on Japanese military targets. People like to argue that the atomic bomb had incredible civilian casualties as compared to the primarily military casualties involved in the island hopping campaign; however, Japanese civilians were dying by the hundred thousand to firebomb attacks. As many Japanese citizens died during a particularly frequent stretch of firebombing in Tokyo as died in the atomic bomb dropped at Hiroshima, and Hiroshima at least stopped the firebombing from continuing for another two or three years. Even considering the massive devastation and loss of human life from the bombs, it seems apparent that this instantaneous loss significantly outweighs the massive casualties that would have been faced in a Soviet invasion or in a massive continuation of the island hopping campaign.

                    The bombing of Nagasaki, however, is an entirely different concern. America had already shown that it had weapons of mass destruction, and certainly did not need to show it again. However, Japan was not ready to give the full unconditional surrender that the leadership wanted. They had implied a conditional surrender allowing them to keep their emperor and their sovereignty. However, the USSR had just invaded from the north without Allied support and the US needed to act quickly, so a second bombing run was authorized. Realistically, Truman probably, at this point, could have continued applying pressure to the southern archipelago, and simply let Japan fall under the threat of atomic bombs, but the Soviets forced his hand. The American leadership did not want the Soviet Union to get any more land than it already had, as they could foresee the post-war world which would be dominated by the Soviet Union and America. Giving the USSR land in Japan forfeited America's domination of the Pacific, which was something to be avoided. In order to force Japan's surrender to the US and not to the joint Allied forces, the US dropped the second bomb. Japan's pacifist party, the ``doves,'' used this highly dramatized event in order to push through an unconditional surrender, which was delivered personally over radio by the emperor.

                       It may seem, initially, that the US had a multitude of options for finalizing the invasion of Japan without resorting to the massive civilian casualties caused by the bomb. However, it appears obvious that the United States was actually able to reduce casualties through the use of the atomic bombs, and ended the war earlier and prevented the Soviets from claiming any portion of Japan.
                   \end{mla}
\end{document}
