\documentclass{article}

\usepackage{ifpdf}
\usepackage{mla}

\begin{document}
\begin{mla}{Lee}{Mracek}{Mary Hill}{American Studies}{\today}{title}
   After Victory in Europe Day, most European nations withdrew from the fighting in World War II to concentrate on recovery after the devastation left as a result of the blitzkrieg and years of bombing. On the Pacific front, however, the United States and the USSR continued to fight violent battle after violent battle in a massive campaign towards the mainland. After the United States occupied Iwo Jima in early 1945, American air support kept Tokyo and other major Japanese cities under constant threat of massive firebomb attacks throughout the entire remainder of the war. In the summer of 1945, all of this would change when the United States developed the atomic bomb and bombed two cities in Japan to force an early surrender.

   In the United States, the options for ending the war in the Pacific seemed limited. Either the US forces would have to island hop all the way to Tokyo, while taking massive casualties and possibly extinguishing millions of Japanese lives, although the leadership did not consider that figure while making a decision. In 1945, however, another option was made available my research done at the top secret Manhattan project: the atomic bomb. The atomic bomb was the single largest destructive device ever developed and seemed like a quick and easy way to end the war in Japan. For Truman, as well as the others who were aware of the project, the decision to drop it was not an easy one. On one hand, Japan had offered no indication of surrender, and the bomb seemed like a simple way to convince the Hirohito and Tojo that the war could no longer be won, or even survived, by Japan. The Japanese warplan at the time was essentially to hold off the US until they could get a favorable surrender, a situation which the presence of the bomb removed. The other option for President Truman was to ask the Soviets to enter into the war in Japan from Hokkaido to draw Japanese reinforcements north rather than into the islands. America, however, wanted to avoid this because it meant possibly having the split the occupation of Japan like they were forced to do in Germany.

   One thing many modern and contemporary sources fail to consider with regards to Truman's decision to drop the bomb is the immense casualties caused by the frequent firebomb attacks on Japanese military targets. People like to argue that the atomic bomb had incredible civilian casualties as compared to the primarily military casualties involved in the island hopping campaign; however, Japanese civilians were dying by the hundred thousand to firebomb attacks. As many Japanese citizens died during a particularly frequent stretch of firebombing in Tokyo as died in the atomic bomb dropped at Hiroshima, and Hiroshima at least stopped the firebombing from continuing for another two or three years. Even considering the massive devastation and loss of human life from the bombs, it seems apparent that this instantaneous loss significantly outweighs the massive casualties that would have been faced in a Soviet invasion or in a massive continuation of the island hopping campaign.

   The bombing of Nagasaki, however, is an entirely different concern. America had already shown that it had weapons of mass destruction, and certainly did not need to show it again. However, Japan was not ready to give the full unconditional surrender that the leadership wanted. They had implied a conditional surrender allowing them to keep their emperor and their sovereignty. However, the USSR had just invaded from the north without Allied support and the US needed to act quickly, so a second bombing run was authorized. Realistically, Truman probably, at this point, could have continued applying pressure to the southern archipelago, and simply let Japan fall under the threat of atomic bombs, but the Soviets forced his hand. The American leadership did not want the Soviet Union to get any more land than it already had, as they could foresee the post-war world which would be dominated by the Soviet Union and America. Giving the USSR land in Japan forfeited America's domination of the Pacific, which was something to be avoided. In order to force Japan's surrender to the US and not to the joint Allied forces, the US dropped the second bomb. Japan's pacifist party, the ``doves,'' used this highly dramatized event in order to push through an unconditional surrender, which was delivered personally over radio by the emperor.

   It may seem, initially, that the US had a multitude of options for finalizing the invasion of Japan without resorting to the massive civilian casualties caused by the bomb. However, it appears obvious that the United States was actually able to reduce casualties through the use of the atomic bombs, and ended the war earlier and prevented the Soviets from claiming any portion of Japan.
\end{mla}
\end{document}
