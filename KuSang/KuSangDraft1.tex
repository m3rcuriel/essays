%          File: Ku Sang Draft 1.tex
%       Created: Mon May 12 09:00 PM 2014 P
% Last Modified: Fri 30 May 2014 07:55:03 AM PDT
%
\documentclass[12pt,letterpaper]{article}
\usepackage{ifpdf}
\usepackage{mla}
\usepackage[english]{babel}
\usepackage[T1]{fontenc}
\usepackage[utf8]{inputenc}
\usepackage[hidelinks]{hyperref}
\usepackage[backend=biber,style=mla]{biblatex}
\usepackage{appendix}
\usepackage{setspace}
\usepackage[right,modulo]{lineno}
\addbibresource{KuSangDraft1.bib}
\nocite{*}
\begin{document}
\title{Peace through Meditation and Religion}
\author{Lee Mracek}
\pagenumbering{gobble}
\maketitle
\pagebreak
\tableofcontents
\pagebreak
\pagenumbering{arabic}
\begin{mla}{Lee}{Mracek}{Cler}{World Literature and Writing}{\today}{Peace through Meditation and Religion}
    \addcontentsline{toc}{section}{Introduction}
\begin{quote}\textit{In spite of everything I still believe that people are really good at heart.}\end{quote} 
\qquad These words were famously written by Anne Frank, a teenage Jewish girl during the Nazi era, as she and her family hid in an attic from the German SS in order to survive the Holocaust. As they hid, they had access to knowledge of the latest horrors committed inside of the German death camps, and it brings a certain maturity to Anne due to her awareness of all of the horrible things perpetuated by humanity. However, despite all this horror, she retains her belief in the goodness
    of humanity as a whole. In our modern, cynical world, this is a hard view to have -- the attrocities of the Cambodian Genocide and the Holocaust have made it harder to believe in an inherent human goodness. However, there are still those who maintain their hope despite first-person experiences with the horrors of humanity. One such person is Ku Sang, a renowned Korean poet, who expresses his hope for humanity through his works throughout his lifetime. Through his works, it is
    posssible to deduce that although humanity is capable of committing horrible attrocities, hope springs from obediance to God and a connection to nature.\\
    \addcontentsline{toc}{section}{Historical Context}
    Ku was born on September 16, 1919 in W\u{o}nsan, a Korean city on the edge of the sea of Japan. He was raised in a devoutly Catholic family, and his brother was a priest. This connection to religon is one that he maintains throughout his life and also one that inspires his poetry.~\parencite{bong-goon} As he approached adulthood, he moved to Japan to study at Nihon University, where he graduated with a degree in Religous Studies. After graduating, he moved back to his hometown
    where he began to publish the poetry that he began writing during college, including the anti-Communist poems ``Path'' and ``Dawning'', published in the collection \textit{Fragrance} in 1946. The Communist Party of Korea was adverse to the publication of these poems, and he was forced to flee the north to South Korea.~\parencite{lti} In South Korea, he worked as a journalist assignd to cover the South Korean military during the Korean War, and later as the editor-in-chief of the
    Yeongnam Daily Newspaper. He later served as a lecturer of poetry at Chungang University, where he continued to publish his poetry. He passed away on May 11, 2005.~\parencite{lti} \\
    \addcontentsline{toc}{section}{Myself}
    One of Ku's first poetic cycles, \textit{Wastelands of Fire}~\parencite{koreanlit} emphasizes his trust and faith in the Will of God~\parencite{bong-goon} during his trials during the preface to the Korean War~\parencite{diego}. The first poem in the cycle, \textit{Myself} deals with Ku's existential reality as he copes with his mortal existance. The poem uses several ornate comparisons of natural phenomenon to conceptualize the enormity of human existance -- the magesty
    of the human soul. Ku spends a lot of time in this poem with religious references. For instance, he refers to ``six senses,'' ``obscure darkness,'' and ``Karma,'' all of which refer to Buddhist traditions. He also references ``seven sins'' and ``Original Sin,'' which spring from the Christian tradition~\parencite{bong-goon}. The ``six senses'' refer to the six sensory organs that exist in the Buddhist tradition -- the eyes, nose, ears, tongue, body, and will. Obscure
    darkness refers to the ignorance of truth -- the oppressing fog of secular worries and a lack of Buddhist values. When he refers to ``seven sins,'' he is referring to the seven sins recognized in Catholic mythos, while ``Original Sin'' is the sin against God committed by Adam and Eve~\parencite{bong-goon}. Also, in the poem, Ku Sang uses short, broken phrases to convey the vastness and incomprehensibility of the human existance. For instance, he says ``the substantiality such fullness gives, / and more
    than its opposing nihility, / more, too, than unknown death'' (36-38). The broken sentences, where an idea is split into multiple sections between commas, represents a slowness of explanations -- he has to think before uttering each word. The frequent use of commas throughout the poem also lends to the overall mood of majesty and awe. Connotations found in the words ``cosmic,'' or ``Eternity'' also help to convey this sense of towering magesty that the poems tries to
    portray. Through ``Myself,'' Ku Sang also appears to be conveying his feelings of his own existentiality and of the many-faceted ideas, including religons that make up his existance~\parencite{lti}.\\ 
    \addcontentsline{toc}{section}{Religions Connections in Christopher's River}
    Ku Sang's intense belief in salvation and hope from God manifests itself in most, if not all of his poems, including in his cycle, ``Christopher's River''~\parencite{bong-goon}. This poetic cycle deals with his reactions and thoughts on the tale of St. Christopher, a saint who carries people across a dangerous river so that they do not have to get wet and dirty. Ku Sang feels that this is the unattainable goal of human perfection -- abject kindness towards all people as guided
    by God~\parencite{churches}. However, in Poem 20 from ``Christopher's River,'' he says ``I have spent today, \ that source of mystery, today, \ wallowing in the dirt.''
    By this, he refers to his sense that his own life and works are just dirt and sewage due to his sins and flaws. The rest of ``Christopher's River'' adds to his explanation of true holiness through contrast with his own life~\parencite{bong-goon}. \\
    Another poem in the cycle that pulls together Ku Sang's thoughts on the struggles of humanity is Poem 16 from ``Christopher's River.'' In Ku Sang's usual style, it is plainly written, and communicates its ideas through enigmatic phrasing and abstract concepts~\parencite{taize}. At first, the poem seems to be beyond the scope of this essay, but it rapidly ties itself to grander themes of humanity with the line, ``The river / unresisting, accepts / every violence, every humiliation, / and
    never denies itself,'' which immediately calls to mind the trials of humanity and the violence that is a part of its history. Using this poem, Ku Sang is advertising the calm route -- the low route that has no conflicts and no violence. He is essentially advertising the Ghandian route to peace~\parencite{taize}. Through this peaceful route, he attempts to express his personal beliefs in the proper behavior of humanity which springs from deep
    reflective thoughts, soul searching, and trust in God.\\
    \addcontentsline{toc}{section}{Humanity's Shame}
    As a virtuous Christian, Ku Sang struggles to not judge someone by external appearances alone, using his religon to cut to the heart of the matter. However, due to this same spirituality, he often finds himself ashamed of the degredation of mankind, believing that mankind has no shame. In ``Shame,'' he uses the image of the shame of animals trapped in a zoo to contrast with the image of people in a city without shame~\parencite{bong-goon}. He describes himself peering ``in search of an animal \
    that knows shame.''This quest takes him on a search in ``the monkey's red hole,'' ``the bear's paw,'' and ``the female parrot's beak.'' He concludes the passage with the phrase, ``I've come to the zoo in search of a \ shame \ long atrophied \ in the people of the city,'' leaving the poem with a pungent reminder that humanity must find its collective shame through God and spirituality, thus allowing the human race to move forwards towards good as a
    collective~\parencite{koreanlit}. \\
    \addcontentsline{toc}{section}{Historical Analysis}
    Ku Sang's beliefs are deeply affected by the historical context of his life, 1919 -- 2004~\parencite{koreanlit}, which traverses the stretch from the general instability following World War I all the way through the rise of fascism and communism, through the Cold War era and the fall of the Berlin Wall, the codification of human rights, the fall of the Soviet Union, the War on Terror, as well as the majority of colonial independence~\parencite{diego}. While Ku Sang was writing
    his major works, the world was jumping between turmoil after turmoil, as well as transitioning from the 'God-fearing' imperialist era to the modern, self-determined world, which helped inspire Ku Sang's belief that religon contains the solutions to the problems of human-kind. To him, the world worked as long as Man followed God, and the turmoil of the transition period helps vindicate his ideas~\parencite{taize}. Others, including interviewee John Mracek, also see this
    historical reality in the works of Ku Sang, saying, ``he was deeply impacted by the Korean War. He endured it, and that shaped his faith. To him, it is his own personal version of Jesus' forty days and forty nights in the desert.'' Ku Sang took this historical events and was able to weave them into a tapestry of ideas about humanity, war, and religion.\\
    \addcontentsline{toc}{section}{This Year}
    One of Sang's works that draws heavily on religon is his poem ``This Year.'' The poem starts off with the phrase, ``As this country rocked like a boat in Galilee's storms,'' connecting it with the most tumultuous part of his life, the Korean War. He also mentions that ``there were many hard things in the family and the world,'' which is also drawing attention to the suffering that he faced during these troublesome times. To cope with these struggles, he states that, ``I spent
    the whole year not losing my belief in God alone,'' which draws interesting parallels to the themes of suffering found in the Bible. John Mracek states that ``he endured [the Korean War] and it gave him faith.'' All of this together draws the portrait of a man struck by a deep religious faith after the suffering that surrounded him, where his only constant was his faith in God~\parencite{diego}.\\
    \addcontentsline{toc}{section}{Afterlife and Eternity} 
    Later in Ku Sang's life, his faith shows in his work through the medium of death. In ``Eternity Today,'' he speaks about the deaths of some of his friends as they all age, as well as his beliefs in the afterlife. Ku Sang begins the poem by informing the reader that ``Today again news came of a friend's death,'' and mentions that ``I hope my turn comes soon.'' This poem is very grim and speaks mainly about his thoughts on death and the afterlife, including his thought that it is
    the dread of what happens after that makes men afraid to die. However, in the later half of the poem we begin to see Ku Sang's views on humanity's struggle. He starts off by arguing that ``the dread of something after death'' is what causes us to fear death, and states that in light of this, his life todays ``is so much amiss.'' Finally, he informs the reader that ``Surely, if I am really concerned about the afterlife, / shouldn't I already begin to live that afterlife.'' What he's
    trying to say is that some fear death because of the life that comes after, but he believes that if one was truly scared of an afterlife, they would begin living it today -- following the rules of the Bible and of God's will so that they may live in Eternity~\parencite{taize}\\
    Through the works of Ku Sang, as well as a deep understanding of his historical context, it is possible to understand his deep beliefs that although humanity is capable of violence and death on an entirely new scale, there is hope that can be found in peace and through deference to the will of God.Ku Sang lived a life filled with spectres of the horrors caused by humanity, but despite this was able to maintain his hope that humanity, at heart, is good, a view that many today will find hard to share.
    \pagebreak
    % \pagenumbering{gobble}
    \addcontentsline{toc}{section}{Works Cited}
    \printbibliography
\end{mla}
\begin{appendix}
    \section{Selected Works}
    \begin{subappendices}
        \singlespacing
        \subsection{Myself}
        \begin{linenumbers*}
It is more than \\
the deep roots of every emotion, \\
big or small, of every kind, \\
that squirm and kick like little children \\
somewhere inside \\\\
and more than \\
the deep-sea fish \\
of six senses and seven sins, \\
that waves its tail \\
like a night-time shadow on a window pane \\\\
more, too, than \\
star-dust littering the yards \\
of Original Sin and Karma, \\
passing through the obscure darkness of the potter's kiln \\\\
and more than \\
the oasis spring gushing from the desert sand, \\
melting again into foam and flowing \\
after filtering through strata of origins and time \\
with their rustle of dry grass, \\
and the crack in the glacier, or even exploding particles \\\\
more, too, than \\
the world, itself smaller \\
than a millet seed \\
in the cosmic vastnesses \\\\
and more than \\
the ether -- fullness of the boundless void \\
reaching beyond billions of light years \\
of starlight \\\\
more, too, than \\
the substantiality such fullness gives, \\
and more than its opposing nihility, \\
more, too, than unknown death \\\\
more, greater, \\
a soundless cosmic shout! \\
An immensity embracing Eternity! \\\\
Myself. \\
        \end{linenumbers*}
        \subsection{20. Christopher's River}
        \begin{linenumbers*}
I have spent today,\\
that source of mystery, today,\\
wallowing in the dirt.\\\\
Along the sewers of my soul,\\
so full of stench and running muck,\\
the spirits of all purity \\
have foamed and died. \\\\
Tomb of Time turned to a muddy slough! \\
Just a trickle of tears flows from the drain \\
and drips into the coal-black stream. \\\\
Sun and moon, and time too, have lost their shine, \\
and all those things that once bloomed flowers of grace \\
reciprocate now with a wilting look. \\\\
Ah! When will that day come \\
when my life and all its meaning \\
will flow into the distant sea \\
and recover eternal freshness? \\
        \end{linenumbers*}
        \subsection{Shame}
        \begin{linenumbers*}
Between the bars and netting wire of \\
Changgy\u{o}ng Gardens’ Zoo \\
I peer, in search of an animal \\
that knows shame. \\
Keeper, I cry! \\
Is there no presage of shame \\
in the monkey’s \\
red hole? \\
What of the bear’s paw, \\
incessantly licked, \\
the whiskers of the seal, \\
the female parrot’s beak, \\
do they betray no harbinger of shame? \\
I’ve come to the zoo in search of a \\
shame \\
long atrophied \\
in the people of this city. \\
        \end{linenumbers*}
        \subsection{This Year}
        \begin{linenumbers*}
As this country rocked like a boat in Galilee's storms, \\
I spent the whole year not losing my belief in God alone, \\
just doing as I could what had to be done. \\\\
Laid up sick, I suffered for more than a month, \\
there were many hard things in the family and the world, \\
but having endured it all meekly, it proved more valuable \\
than any good fortune could have been. \\\\
These days, as I dream bright dreams of the world beyond, \\
entrusting all things to His divine Will, \\
even if storms are forecast for the coming New Year \\
there is nothing I fear. \\
        \end{linenumbers*}
        \subsection{Eternity Today}
        \begin{linenumbers*}
Today again news came of a friend's death. \\
Well, we all have to go, \\
some sooner some later. \\\\
I hope my turn comes soon.\\\\
Is it fear of the pain before we die \\
that makes death so threatening? \\
Surely there is always euthanasia? \\\\
But the dread of something after death \\
makes that a problem too. \\
The lights and darks of that other world.\\\\
While I evoke in this way \\
the afterlife, my life today \\
is so much amiss.\\\\
Surely, if I am really concerned about the afterlife,\\
shouldn't I already begin to live that afterlife,\\
or rather, Eternity,\\
today? \\
        \end{linenumbers*}
        \subsection{16. Christopher's River}
        \begin{linenumbers*}
The river \\
continues the past, \\
is not imprisoned by the past. \\\\
The river, \\
while living today \\
lives the future too. \\\\
The river, \\
though innumerably collective, \\
keeps unity and equality. \\\\
The river \\
makes itself an empty mirror \\
in which all things view themselves. \\\\
The river \\
at all times and in all places \\
chooses the lowest place. \\\\
The river, \\
unresisting, accepts \\
every violence, every humiliation, \\
and never denies itself. \\\\
The river \\
gives freely to all that lives \\
and looks for nothing in return. \\\\
The river \\
is its own master, \\
free despite all bonds. \\\\
The river, \\
caught between generation and extinction, \\
reveals Eternity within impermanence. \\\\
The river \\
every day in its Pantomime \\
teaches me many things. \\
        \end{linenumbers*}
    \end{subappendices}
    \section{Interview Transcript}
    \begin{subappendices}
\textbf{Lee Mracek:} Hello, thank you for agreeing to this interview. Are you okay being recorded? \\
\textbf{John Mracek:} Sure. \\
\textbf{Lee Mracek:} Okay, so I'll start us off. So John, Ku Sang lived through a very tumultuous time, including a World War, the Korean War, and the rise and fall of the Soviet Union. What influences of these events do you see in his poems?\\
\textbf{John Mracek:} It seems that from his poems, he was deeply impacted by the Korean War. He endured it, and that shaped his faith. To him, it is his own personal version of Jesus' forty days and forty nights in the desert; enduring his hardship meekly. He is also very clear in his poetry with regards to what he is talking about, even explicitly mentioning the Korean War.\\
\textbf{Lee Mracek:} Yeah, I see what you mean. What do you think is his opinion on humanity's departure and distance from God?\\
\textbf{John Mracek:} Like the lack of religion in culture? \\
\textbf{Lee Mracek:} Yes. \\
\textbf{John Mracek:} Well he certainly isn't happy about it. He looks to nature for inspiration, and when he condemns himself within creation, he is condemning mankind for not knowing what all the living creatures in nature know. \\
\textbf{John Mracek:} Humanity has lost touch. What is interesting is that he is a crucible or microcosm of humanity. He does not criticize humanity explicitly, but only by proxy through questioning himself. In that regard he is humble. \\
\textbf{Lee Mracek:} I see. \\
\textbf{Lee Mracek:} What solution for this do you think he encourages in his works? \\
\textbf{John Mracek:} Thinking and meditation, clearly. The whole rock thing. \\
\textbf{Lee Mracek:} So you could say that he thinks that we need to develop a closer connection to God? \\
\textbf{John Mracek:} He points to the divine that he sees in himself as an example to others. Every person has eternity within them as they embrace God.\\
\textbf{Lee Mracek:} Okay. Thank you for your time. \\
\end{subappendices}
\end{appendix}
\end{document}
