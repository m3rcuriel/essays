\documentclass[letterpaper, 11pt]{article}

\usepackage{ifpdf}
\usepackage{mla}
\usepackage[utf8]{inputenc}

\begin{document}
\begin{mla}{Lee}{Mracek}{Mary Hill}{American Studies}{\today}{\textit{The Things They Carried}, 89-123}
Mary Anne, the girlfriend of a soldier, utters this passage while attempting to justify her night excursions and strange affinity with combat in Vietnam to the soldiers who worry for her stability.

The passage attempts to impress on the soldiers that their ``little fortress'' does not accurately reflect the real war. Vietnam, on the other hand, to her seems a vast ephemeral struggle in which she had found ``exactly who [she is].'' The endorphins and adrenaline of combat call to her and she feels as if she ``is glowing in the dark'' from the sheer energy and excitement that she believes can only be found in Vietnam, out in the wild. The country entices her, making her ``want to swallow the whole country'' from her intense hunger for the intense feelings that she now only feels through combat. In war, anyone can become ``one of the animals.'' Mary Anne's allusions to wanting to ``eat'' the whole country, ``the dirt, the death'' inspire images of a primal beast, corrupted by the Vietnam War. Stalking is the one place where she truly knows where she is. Through the passage, we see the seductive allure of war and the rousing effect it can have to transform a cheerleader into a Green Beret with a necklace of tongues. The war transforms Mary Anne and turns her into a primal beast at home in a war-zone. War has that effect on people, even innocent cheerleaders fresh out of high school, and transforms what was once a true ``American girl'' into a primal beast with an appetite for for the entire country, or the entire feeling of Vietnam, and become one with the country and its violence. % goddamn it this is already longer than what I had.
Her manner of speaking, even, is both fervent and rationalizing. Mary Anne is speaking fanatically about her obsession for the conflict in Vietnam and for the feelings that being out at night in combat and in danger elicit in her. Unlike she has ever felt before, she feels ``close to [her] own body.'' She feels more connected to herself than she has anywhere else because that is what war does. It pulls and calls and reaches out to that primal beast that Mary Anne and many others have and brings it out until, like Mary Anne says, ``you can't feel that anywhere else.'' The beast in Mary Anne causes her to become essentially addicted to war because she is unable to feel that sort of pressure or adrenaline anywhere else.
\end{mla}
\end{document}
