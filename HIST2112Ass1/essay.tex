\documentclass[12pt, letterpaper]{article}

\usepackage[english]{babel}
\usepackage[T1]{fontenc}
\usepackage[utf8]{inputenc}
\usepackage[super]{nth}
\usepackage{ifpdf}
\usepackage{mla}
\usepackage[backend=biber,style=mla]{biblatex}

\addbibresource{essay.bib}
\nocite{*}

\begin{document}
\begin{mla}{Lee}{Mracek}{Amsterdam}{HIST 2112}{\today}{Progressives, Psychology, and Poles}
  The Progressivism movement at the beginning of the \nth{20} century was
  defined by a belief that the men were inherently malleable and thus the nature
  of their environment had considerable effect upon their actions and psyche~\parencite{psych}.
  This ``environmental theory of human behavior'' explained that the world that
  people would encounter every day would strongly reflect in how they viewed the
  world. This idea calls back to earlier writers like Josiah Strong, who
  proposed that the ills of America as a whole were in a large part caused by
  the ills of the city on a much smaller level~\parencite{strong}.

  The Progressives largely accepted this view because it fit in with the
  movements overall goals -- to change the environment and thus change the
  population and country as a whole. Progressives did not aim to throw every
  evil-doer in prison, but instead sought to reform the corruption and
  lawlessness which created the ruffians in the first place. In this sense, it
  follows that they quickly adopted the idea that the environment heavily
  impacts the psyche. After all, without that fundamental scientific tenet, the
  movement would essentially have no justification to continue to change the
  city environment to provide a more healthy place for the collective American
  mind~\parencite{psych}. 

  Czolgosz, then, became the perfect example of this environmental theory of
  behavior. Progressives and psychologists argued that his early life molded him
  into the anarchist and general drag on society he became. Czolgosz moved to
  Detroit when he was five, quickly introducing him to the mental toxins of
  urban life. He found work in a glass factory, and later at a mill company. The
  economic crash at the end of the \nth{19} century caused a strike and wrecked
  Leon's job~\parencite{mckinley}. Proponents of the environmental theory, then, would argue that the
  combination of urban lifestyle and poverty caused him to become clinically
  insane and tragically end the life of President McKinley. However, the theory
  fails to address the real poverty and corruption that Czolgosz did encounter
  during his unemployment, and fails to link his assassination of William
  McKinley to any political motives. In effect, the theory attempts to link his
  insanity with his upbringing, when in reality his upbringing was shared by
  many in the late \nth{19} century, but his political radicalism was not.
  \pagebreak
  \printbibliography
\end{mla}
\end{document}
