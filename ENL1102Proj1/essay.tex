\documentclass[12pt, letterpaper]{article}

\usepackage[T1]{fontenc}
\usepackage[utf8]{inputenc}
\usepackage{ifpdf}
\usepackage{mla}

\begin{document} \begin{mla}{Lee}{Mracek}{Afflerbach}{English
1102}{\today}{Deephaven -- An Isolate Inflationary Zone}
In cosmology, scientists accept as a given that the universe we inhabit began
infinitely small, and inflated at differing rates to reach its unfathomably
large present state. To be clear, inflationary theory does not tell us that
stars and galaxies grow ever further apart until we're off hurtling along in our
galaxy, on our own. Instead, space-time itself expands. Instead of the gaps
between objects growing larger, space itself stretches, as if we live on the
skin of an inflating balloon. Some proponents of this theory contend that our
universe could simply be one of many inflationary zones: growing in parallel and
never interacting. In the story ``Deephaven Excusions'', by Sarah Jewett,
the reader begins to understand Deephaven as analogous to one of these isolate
universes. Through her evocative descriptions of scenery and carefully measured
interactions with the denizens of Deephaven, Jewett shows a universe growing and
evolving in parallel with the outside world, soon to be left behind.

From the specific phrasing of Jewett's descriptions of locations, she shows us a
Deephaven fading physically while also crumbling on a more spiritual level.
When Kate and Helen travel to East Parish, a town Deephaven considers duller
than itself, they find the same sort of slow decay also present in Deephaven.
Coming upon an old house, they immediately note that it had fallen into ruin,
and was an example of ``old-fashioned workmanship'', with a fence that had been
imposing and impressive ``in its day''. However, Jewett immediately contrasts
this splendor by explaining that now, the fence ``could barely stand alone.''
Much like Deephaven and the New England fishing environment itself, these once
great towns remain impressive, but in the old-fashioned sense, and no longer
``stand alone'' among the growing tapestry of a culturally unified America.
Continuing in the same description, the ``great'' chimneys were of fancy bricks
``which used to be brought from over-seas.'' The past tense wording ``used to be
brought'' as well as the contrast of these once-bright red bricks and their
current faded form force the reader to come to terms with the decay of the town
itself. After all, the towns, much like the bricks, were established in grandeur
``in the days of the colonies,'' which in context sounds as far away as the
Pyramids in Egypt but in reality was only 150 years prior (285). The faded and
unmaintained house cannot help but be overwritten and inevitably fall by the
wayside as the arm of progress moves onwards.  Much like the house, Deephaven
and its immediate environs have fallen off from mainstream America and are now
merely a façade which will be remarked upon by these two passing rich girls from
the city, but remain intrinsically unremarkable on a larger scale.

No character in Jewett's story parallels the destruction of Deephaven's
uniqueness more than the character and life of Sally Chauncey. She acts as
Jewett's conduit to express the way Deephaven's denizens who were around when
Deephaven was wealthier cling to the illusion of grandeur. However those
illusions cannot last as time continues to pass. Like Deephaven itself, in her
youth Sally lived a life of wealth and happiness, embodied in the wealthy old
mansion she still tries to live in. In the old house, ``she lives mostly in the
past,'' which parallels the way that Deephaven lives mostly in the past.
Furthermore, she refuses to comprehend the fact that many of the families she
once knew quite well have since passed away. When Kate informs Sally that Kate's
aunt recently passed away, Ms. Chauncey replies, ``\,`Ah they say everyone is
'dead' nowadays.'\,'' She then goes on to proclaim that this whole situation is a
``silly idea'' that she cannot ``comprehend'' (287). This lack of understanding
which Jewitt portrays through the specific use of ``comprehend'' points to the
lack of understanding that the people of Deephaven have with respect to the
inevitable loss of their bubble. To them, it remains a ``silly idea'' that any
of these pockets of American culture will ever be overwritten by a mainstream
norm, when in fact that result is effectively inevitable. Unfortunately, much
like Deephaven itself, Sally comes to a rather ignoble end. Unable to accept the
loss of her status and of her house, she returns, and had to ``wade though
half-frozen water'' (289) in order to get in. The comparison between the frozen water
she waded through and the frigid surf around Deephaven seems appropriate, as
this cold is what leads to her dead from illness. In the end, her inability to
accept the present causes her death, in the old house she could not let go of.


Equally hauntingly, the funeral which Helen and Kate attend frames a funeral for
Deephaven itself, perhaps not so far away and just as sudden as that in the
story. The man the town is mourning had fallen into decay and inevitably passed
away, and on the day of the funeral Jewett does her best to describe a dismal
New England day, complete with cold air, wind, sea cliffs, and monotonous white
breakers. The funeral itself reinforces this dismal atmosphere, with Helen
observing that ``the mourners looked so few.'' Even the funeral wagon has seen
some decay, as Jewett specifically highlights the ``rattle of the
wagon-wheels.'' More pointedly, one of the witnesses asks plaintively, ``\,`He's
gone, an't he?'\,'' The narrator immediately echoes this concept, telling the
reader ``that was it --- \textit{gone}'' (284). Between the rhetorical question
asked by the witness and the narrators reiteration of the weight of the simple
concept 'gone,' Jewett evokes in the reader the same feeling the mourners must
be feeling. The descriptions forces the thought that here is a man who each of
these mourners had known and had worked with, and then suddenly he was simply no
longer there. There was no long-term illness, and no justice in his death, it
simply happened. Thus Jewett brings into consideration \textit{Deephaven's}
funeral. Few will notice when Deephaven loses whatever cultural legacy and
uniqueness it originally had, and fewer still would attend its funeral, imagined
to be complete with its own rattly wagon carting it's corpse away from the new
cultural centers of the developing nation. From this stark scene, Jewett draws
the parallels between the decay of a man whose wife has died, and the decay of
the town whose livelihood has passed away.

Between the stark New England environment and the dilapidated town, Jewett
paints a portrait of an isolated bubble of America, attacked on all sides by the
growing urbanization and uniformity of the nation. Unfortunately, Jewett also
forces the reader to come to terms with this decay as inevitable, and instead
attempts to produce, like Kate and Helen, another onlooker for Deephaven's final
fall from grace and eventual absorption into a new continuum. Much like a
physically bounded universe, Deephaven and similar towns form a cultural bubble
in the American landscape which will inevitably be consumed and blend into the
uniform expansion of the rest of the nation's culture and diversity.
\end{mla}
\end{document}
