%        File: iat.tex
%     Created: Fri Mar 06 10:00 AM 2015 P
% Last Change: Fri Mar 06 10:00 AM 2015 P
%

\documentclass[a4paper, 11pt]{article}
\usepackage{ifpdf}
\usepackage{mla}
\usepackage[T1]{fontenc}
\usepackage[utf8]{inputenc}
\begin{document}
\begin{mla}{Lee}{Mracek}{Hill}{AP Am. Studs}{\today}{IAT Reflection}
    Taking the Harvard IATs formed an interesting reflection on my own internal biases which I may not have even known about. I took three tests, Asian American, Gay-Straight, and the Obama-Bush test. The results were rather surprising and in some ways contrary to what I had indicated as my expected preference. I even began to question either the accuracy of the test or the presence of other biases. \\
    On my first test, I discovered a strong preference to view Asian Americans as more ``American'' than their European-looking counterparts. This result made the most sense of the three. Because I live in Cupertino, I am exposed to a large percentage of Asian-looking individuals and as a result identify them as American. The only European looking individuals that I am exposed to are actually first or second generation immigrants from locations such as Germany or Russia, so it makes sense that I have an unconscious association of foreign with European and American with Asian, and the test corroborated that. It will be interesting, however, to see what happens once I move outside of the Cupertino bubble and am exposed to the rest of America where large percentages of Asian populations do not exist. I do not think, however, that this bias has much bearing on my day to day decisions. \\
    The next test I took checked for bias between homosexual and heterosexual individuals. The result was more shocking, and also more annoying. According to the test, I have a moderate preference for the heterosexual over the homosexual. This does not really surprise me. I am (probably) heterosexual, and have been exposed my entire life to anti-homosexual media, as well as the opinions of others. The test results, while they do annoy me (who wants to be biased?), also reaffirm my belief that I have removed some of the most destructive personal biases which I have. It makes me happy that while I have this implicit bias during this test, I consciously am able to go against or reverse it given time to apply my conscious thinking mind to an issue. \\
    The final test that I took was by far the most surprising. I apparently exhibit a \textbf{strong} preference towards George W. Bush over Barack Obama. I do not really even know what to say to that. It is at once both shocking and repulsive, because much as I am ambivalent towards George W. Bush as a person, I feel only distaste for his policies, his term, and his decisions regarding the nation and its people. While I can understand, and even agree with some of the things that he pushed through Congress during his eight-year rule, I think that as a leader he was an awful choice. As a result, I fail to understand how I got this result. This makes me question the exam as a whole, but I also recognize that I could have an underlying bias for white against black or old against young that can be interpreted as a preference for George W. Bush, although I am still rather confused. I really did think that I found George W. Bush distasteful. I went into the test thinking that this would be a no-brainer, and interestingly it was not.\\
    My foray into subconscious biases helped me to understand the brainwaves which govern my life, but also understand how those who have not been as educated regarding these biases can simply not even acknowledge them. If I were making a decision on gay marriage and I had not had the personal experience that I have, I may have gone against the idea solely on the bases of my unconscious bias. Unfortunately, we cannot do much to avert these biases, as they are intrinsic within human psychology.
\end{mla}
\end{document}
