\documentclass[letterpaper, 11pt]{article}

\usepackage{ifpdf}
\usepackage{mla}
\usepackage[T1]{fontenc}
\usepackage[utf8]{inputenc}
\usepackage[style=mla,backend=biber]{biblatex}

\bibliography{mlabib}

\begin{document}
\begin{mla}{Lee}{Mracek}{Mary Hill}{American Studies}{\today}{The Emotional Truth in Storytelling}
    Since the dawn of time, we have been storytellers. Human language, human culture, and human society all arose out of oral traditions passed on through stories far before memories became writing. In order to be memorable on the human scale, stories had to carry emotion. Bland stories lack the power required to transfer meaningful information from one generation to the next, while emotions allow those stories to carry meaning intrinsic to oral tradition. Because of the necessity of emotional truth in storytelling, Tim O'Brian's \textit{The Things They Carried} proves that stories need not be true if they can convey an emotional impact. O'Brian demonstrates this through both his out-of-character analysis and through the stories told by some of his characters.

    O'Brian, in his own story, ``How To Tell A True War Story'' about truth in war stories, explains that true war stories are meant to reflect ideas and feelings, and not necessarily reality or moral right. The truth, as a matter of fact, fails to carry a story's meaning in full because it lacks that emotional truth that good storytelling requires in order to captivate and teach an audience. Without that empathetic connection, stories lose meaning. With regards to true war stories, O'Brian states, ``you can tell a true war story by its absolute and uncompromising allegiance to obscenity and evil'' \cite[69]{TTTC}. A \textit{true} war story, in O'Brian's eyes requires an element of ``evil'' and a lack of moral meaning in order to adequately convey its greater emotional truth. Much like reality, a war story does not have some greater meaning, or lesson to teach. It simply happened. A true war story is ``absolute and uncompromising,'' it does not give anything to the reader because in doing so it compromises the devotion to emotional truth which a true war story requires. Furthermore, O'Brian's use of the words ``obscenity'' and ``evil'' imply a degree of ugliness or unpleasantness that a war story leaves with its audience in order to hit them hard with an emotional truth that cannot be captured by fairytale or moral. In war stories, ``Often the crazy stuff is true and the normal stuff isn't, because the normal stuff is necessary to make you believe the true incredible craziness'' \cite[71]{TTTC}. True war stories must be diluted with ``normal stuff,'' because if certain material is not fabricated, then the audience will not believe the emotions conveyed by the story in its intrinsic craziness.The tone of both of these passages talking about criteria for war stories is direct and authoritative. O'Brian is not reminiscing or arguing. He states, based on his own experience with storytelling and emotional truth, that war stories do not have morals and war stories do not necessarily reflect an objective reality in order to accurately convey the emotional truth intended by the storyteller.

    In one of O'Brian's stories, Rat Kiley, a frequent liar and embellisher, explains to O'Brian that he exaggerates stories in order to convey their emotional truth. Without exaggeration, his stories would fall flat because they lack the emotional flair that real life possesses. For Rat, ``It wasn't a question of deceit. Just the opposite: he wanted to heat up the truth, to make it burn so hot that you would feel exactly what he felt'' \cite[89]{TTTC}. The key to storytelling, from Rat's eyes as a master storyteller, is so that ``you would feel exactly what he felt.'' Storytelling has nothing to do with conveying objective truth, merely with making the audience understand the emotions of the storyteller. Stories are only enrapturing or instructional if they ``burn'' emotionally and captivate the audience, which is what Rat Kiley tries to enable his stories to do. For Rat, and O'Brian, ``facts were formed by sensation, not the other way around'' \cite[89]{TTTC}. This ``sensation'' that Rat Kiley mentions is obviously not something objective, or something ``true,'' because sensations or gut feelings rarely are.

    One of Rat Kiley's more notable stories is the story of Mary Anne, and while Mary Anne's story may not be true, it is the only way to capture the transformation elicited by war. Mary Anne's story, at first, seems like utter fiction, but it is rapidly apparent that a story as ridiculous as Mary Anne's is the only way to capture the transformation elicited by war on the minds of the soldiers. War transforms men and women into beasts, and the best way to capture that is in the innocent Mary Anne's transformation into a Green Beret fighting beast. The repulsion the reader feels at Mary Anne's transformation carries the repulsion that the soldiers such as Rat Kiley must have felt at their own transformation from mortal men into beasts of war. Mary Anne, while justifying her late night excursions, tells the soldiers, ``I want to swallow the whole country---the dirt, the death---I just want to eat it and have it there inside me'' \cite[111]{TTTC}. Using words such as ``swallow,'' and ``eat,'' O'Brian portrays Mary Anne as some dark beast through his diction. She wants to consume the violence in Vietnam, and she's addicted to the adrenaline and endorphins that she can get in Vietnam but not in cheerleading back at Cleveland Heights. The story itself does not have to objectively represent what happened, but instead must pass along that emotional transformation from innocent high school girl into beast in Vietnam. As Rat's telling of Mary Anne's story represents, the actual truth of a story is irrelevant alongside the emotional truth that Rat and O'Brian are attempting to convey regarding the emotional and psychological shift faced by the soldiers in Vietnam. Mary Anne is just an allegory for the life of a soldier that allows the storyteller to convey the proper emotional truth.

    Like all great storytellers, both O'Brian and the storytellers in his book recognize that emotional truth is far more important to a stories meaning and impact than the objective truth. Boring and bland stories lack the emotional hook that any narrative form uses to hook the audience into the truth of the story. Emotional, exaggerated storytelling, as done by O'Brian, is the only way to convey the emotional truth of a story, even if it does not convey the objective truth of the story. Without the exaggerated style of storytelling, human culture would be much more bland because everyone would attempt to pass along facts instead of a deeper meaning or truth that can only be captured by exaggeration and by emotional reality.
\clearpage
\printbibliography
\end{mla}
\end{document}
