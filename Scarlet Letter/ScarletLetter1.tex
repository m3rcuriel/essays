%        File: Scarlet Letter 1.tex
%     Created: Tue Nov 25 10:00 PM 2014 P
% Last Change: Tue Nov 25 10:00 PM 2014 P
%
\documentclass[12pt, letterpaper]{article}
\usepackage{ifpdf}
\usepackage{mla}
\usepackage[utf8]{inputenc}
\usepackage[english]{babel}
\usepackage[super]{nth}

\pagenumbering{arabic}
\begin{document}
\begin{mla}{Lee}{Mracek}{Hill}{American Studies}{\today}{Utopian Commentary in the Scarlet Letter}
\begin{quotation}
\textit{We the People of the United States, in Order to form a more perfect Union\ldots do ordain and establish this Constitution for the United States of America.} --- The Preamble to the Constitution of the United States of America
\end{quotation}
The great ``American Experiment,'' the creation of a country run by its people, with equality for all, can in many ways be considered an echo of an intrinsic human need to build the perfect society or utopia---a ``more perfect Union.'' From examples throughout history, however, it becomes obvious that all attempts thus far to create a utopia have resulted instead in a oppressive dystopia in which the legal system becomes rule of the ``correct,'' be they religious leaders or social zealots. In \textit{The Scarlet Letter}, Hawthorne uses a caricature of Puritan society to convey his conclusions on the idea of utopia. Puritan society was originally envisioned as a way to become closer to God, and to create a utopia in the New World. However, Hawthorne uses the character arcs in \textit{The Scarlet Letter} to assert that Puritan ideals are an abomination of human spirit, and instead create dystopia. \\
        The primary conflict between a character and Puritan society is Hester's public sin versus the repression and secrecy that the Puritan ideology condones. Hawthorne's primary way of expressing this is through symbolism in the descriptions of Hester and in her interactions with the environment. When Hester Prynne and Pearl are walking in the forest, Pearl remarks to Hester that, ``the sunshine does not love you. It runs away and hides itself because it is afraid of something on your bosom \ldots I am but a child. It will not flee from me; for I wear nothing on my bosom yet'' (180).
        From Pearl's statement, it can be understood that even sunshine---light itself---withdraws from Hester because of the letter that she wears. However, sunlight is natural light. The symbolism in this statement comes not from the light withdrawing from Hester, but from nature withdrawing from the 'A.' The 'A' on Hester's breast is not just a symbol of her adultery---it is a symbol of the society that condemns her, and using the sunlight Hawthorne attempts to portray nature as aligned against Hester. Another use of symbolism to portray nature versus society through Hester Prynne is in the single rosebush by the entrance of the jail in which she is held. In the barren lot that contains the jail, ``on one side of the portal, and rooted almost at the threshold was a wild rosebush \ldots in token that the deep heart of Nature could pity and be kind to [the criminal]'' (46). Society declared this tract of land to be their jail, causing it to become desolate and forsaken. In the midst of the blight of society, however, grows a single aspect of nature which provides hope to the people who have been condemned by society, but not by nature. The rosebush is nature's way of telling the Puritan's that their laws and social ideas contradict the ideas of the far more forgiving benevolence of nature, as well as contradicting the quintessence of humanity. If \nth{17} century Boston was truly a religious utopia, people would be encouraged to act kindly for the good of their society. However, the Puritan leadership ostracizes Hester for her sin, and thus deride any good that she does as tainted by darkness. After she is exiled, `she becomes proficient in needlework, and ``her needlework was seen on the ruff of the Governor, military men wore it on their scarfs, and the minister on his band; it decked the baby's little cap; it was shut up to be mildewed and moulder away, in the coffin of the dead. But it is not recorded that, in a single instance, her skill was called in aid to embroider the white veil which was too cover the pure blushes of a bride. The exception indicated the ever relentless vigor with which society frowned upon her sin'' (80). Additionally, ``Hester bestowed all her superfluous means in charity, on wretches less miserable than herself\ldots Much of the time which she might readily applied to the 
        better efforts of her art, she employed in making course garments for the poor'' (80). Not only has Hester become a renowned seamstress whose needlework is featured at virtually every level in Puritan society, worn ``on the ruff of the Governor,'' or by ``the minister on his band,'' she also donates most if not all of her spare time to ``making course garments for the poor,'' who, by Hawthorne's word as a narrator can be considered ``less miserable than herself.'' Her charity, the widespread use of her needlework, even her choice to remain in Boston rather than flee her shame all point to an incredibly strong, talented and caring woman who goes out of her way to help those less fortunate than her, or even those \textit{more} fortunate than her. In essence, based on textual evidence, it appears as though Hawthorne considers Hester to be a ``good'' person - morally sound and caring. Even though it seems as though Hester's acts may serve to completely outweigh a relatively small misdemeanor such as adultery, the town refuses to remove their judgment. Hawthorne remarks that ``in all [Hester's] intercourse with society\ldots 
        Every gesture, every word, and even the silence of those with whom she came into contact, implied, and often expressed that she was banished\ldots'' (81). All of this good that Hester Prynne performs is lost to the larger view of society because they have so thoroughly stigmatized her as a sinner that any good she does is already tainted by the Hell, and everyone conveyed to her the truth that ``she was banished,'' both from the social circles of Boston, but also from the greater goals of society. By attempting to purge sin from the inhabitants of Boston, the Puritans have instead removed the weight of the good that Hester Prynne did, and prevented any stigmata from being removed from her charity work. Through this, Hawthorne presents his views on Puritan society as a dystopia rather than a utopia. \\
        Throughout the novel, Pearl acts as a natural counterbalance to society, and the oppression of her natural opinion sets up the conflict of Puritans versus nature. Pearl was born in exile from Puritan society, and as such is raised without the unnatural structures that the religious leadership built in their society to suppress human nature. As a result of this discrete existence, Pearl and the natural world have a connection that has been untainted by the constructs of a legalistic society. While the sunlight avoids Hester, Hester perceives that ``Pearl did actually catch the sunshine, and stood laughing in the midst of it it, all brightened by it's splendor, and scintillating with the vivacity excited by rapid motion \ldots until her mother had drawn almost nigh enough to step into the magic circle too'' (180). Pearl is disjointed from society and is instead tied to nature, which allows the sunlight to come to her instead of her mother. Surrounded by nature, whose love is expressed by the sunlight beaming down on Pearl, she cannot help but be brightened and vivacious. Hawthorne uses his elaborate imagery to portray Pearl as uplifted and ``brightened'' by the influence of nature. The Puritan society, however, acts in direct contrast to the gentle descriptions offered by Hawthorne's narration. When Pearl is at the governor's hall to be judged, the governor exclaims that ``\,`Here is a child of three years old, and she cannot tell who made her! Without question, she is equally in the dark as to her soul, its present depravity, and future destiny! Methinks, gentlemen, we need inquire no further'\,'' (109). Without any specific reason, the Puritans condemn Pearl as an abomination, a devil child, simply because she does not coincide with existing Puritan beliefs. Because she does not understand their culture, she cannot understand her soul's ``present depravity,'' implying that she is already full of sin. How, though, can someone who is so full of sin be characterized as ``brightened and vivacious'' under beams of sunlight out in the forest. Hawthorne describes Pearl in the forest as aglow: lit from within by sunlight. The Puritans, however, do not respect her inner natural light, and thus go against what Hawthorne thinks of as natural good. In his view, Puritan values directly contrast with natural human spirit, and cause depression and stagnation.
        The characters Hester Prynne, Pearl, and Roger Chillingworth all serve to portray the inevitability of the failure of utopia. Their fall from grace, or oppression by society all point towards an inherent failure within the idea of a ``perfect society.'' Throughout history, humanity has attempted to reach that goal, but perhaps, it is a dream that can never be realized.
\center{\textsc{Work in Progress}}
\end{mla}
\end{document}
