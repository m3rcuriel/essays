%        File: Scarlet Letter 1.tex
%     Created: Tue Nov 25 10:00 PM 2014 P
% Last Change: Tue Nov 25 10:00 PM 2014 P
%
\documentclass[12pt, letterpaper]{article}
\usepackage{ifpdf}
\usepackage{mla}
\usepackage[utf8]{inputenc}
\usepackage[english]{babel}
\usepackage[super]{nth}
\usepackage[backend=biber, style=mla]{biblatex}

\pagenumbering{arabic}
\begin{document}
\begin{mla}{Lee}{Mracek}{Hill}{American Studies}{\today}{Utopian Commentary in the Scarlet Letter}
\begin{quotation}
\textit{We the People of the United States, in Order to form a more perfect Union\ldots do ordain and establish this Constitution for the United States of America.} -- The Preamble to the Constitution of the United States of America
\end{quotation}
The great ``American Experiment,'' the creation of a country run by its people, with equality for all, can in many ways be considered an echo of an intrinsic human need to create a ``more perfect Union'' -- to build the perfect society or utopia. From examples throughout history, however, it becomes obvious that all attempts thus far to create a utopia have resulted instead in a oppressive dystopia in which the legal system becomes rule of the ``correct,'' be they religious leaders or social zealots. In \textit{The Scarlet Letter}, Hawthorne uses a caricature of Puritan society to convey his conclusions on the idea of utopia. Puritan society was originally envisioned as a way to become closer to God, and to create a utopia in the New World. However, Hawthorne uses the Puritan society in the Scarlet Letter to assert that Puritan ideals are an abomination of human spirit, and instead create dystopia. \\
\center{WORK IN PROGRESS}
\end{mla}
\end{document}
