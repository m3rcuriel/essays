%        File: Scarlet Letter 1.tex
%     Created: Tue Nov 25 10:00 PM 2014 P
% Last Change: Tue Nov 25 10:00 PM 2014 P
%
\documentclass[12pt, letterpaper]{article}
\usepackage{ifpdf}
\usepackage{mla}
\usepackage{setspace}
\usepackage[utf8]{inputenc}
\usepackage[english]{babel}
\usepackage[super]{nth}
\usepackage[backend=biber, style=mla]{biblatex}

\pagenumbering{arabic}
\singlespacing
\begin{document}
\begin{mla}{Lee}{Mracek}{Hill}{American Studies}{\today}{Utopian Commentary in the Scarlet Letter}
\doublespacing
\begin{quotation}
\textit{We the People of the United States, in Order to form a more perfect Union\ldots do ordain and establish this Constitution for the United States of America.} -- The Preamble to the Constitution of the United States of America
\end{quotation}
The great ``American Experiment,'' the creation of a country run by its people, with equality for all, can in many ways be consider in many ways an echo of a base human need to create a ``more perfect Union'' - to build the perfect society. In other words, a utopia. From examples throughout history, however, it becomes obvious that all attempts thus far to create a utopia have resulted instead in a oppressive dystopia in which the legal system becomes rule of the ``correct,'' be they religious leaders or social zealots. In \textit{The Scarlet Letter}, Nathaniel Hawthorne uses a caricature of \nth{17} century Puritan society to protest and denounce the assorted utopic social movements that took place in the \nth{19} century during his life. Through the harrowing journey of Hawthorne's characters, he paints a picture of overall ``good'' men, women, and children, that have been destroyed by the hypocrisy of a society that stifles their human values while preaching hypocritical values at them. The struggles of Hester Prynne, Roger Chillingworth, Arthur Dimmesdale, and Pearl, convey the message that instead of creating utopia, the movements of both the \nth{17} and \nth{19} centuries instead created dystopia. \\
\textit{The Scarlet Letter} is in many ways the tale of the personal redemption of Hester Prynne in her quest to do good for her fellow humans against the condemnation of her society. When her adultery is discovered, she is immediately branded as an evil sinner by the leaders of the \nth{17} century Boston, any good she does is immediately tainted by an irrevocable stain of sin and darkness.\\
When Hester is released to return to the outside world, she becomes proficient in needlework, and ``her needlework was seen on the ruff of the Governor, military men wore it on their scarfs, and the minister on his band; it decked the baby's little cap; it was shut up to be mildewed and moulder away, in the coffin of the dead. But it is not recorded that, in a single instance, her skill was called in aid to embroider the white veil which was too cover the pure blushes of a bride. The exception indicated the ever relentless vigor with which society frowned upon her sin'' (80). Additionally, ``Hester bestowed all her superfluous means in charity, on wretches less miserable than herself\ldots Much of the time which she might readily applied to the better efforts of her art, she employed in making course garments for the poor'' (80). Not only has Hester become a renowned seamstress whose needlework is featured at virtually every level in Puritan society, worn ``on the ruff of the Governor,'' or by ``the minister on his band,'' she also donates most if not all of her spare time to ``making course garments for the poor,'' who, by Hawthorne's word as a narrator can be considered ``less miserable than herself.'' Her charity, the widespread use of her needlework, even her choice to remain in Boston rather than flee her shame all point to an incredibly strong, talented and caring woman who goes out of her way to help those less fortunate than her, or even those \textit{more} fortunate than her. In essence, based on textual evidence, it appears as though Hawthorne considers Hester to be a ``good'' person - morally sound and caring. Even though it seems as though Hester's acts may serve to completely outweigh a relatively small misdemeanor such as adultery, the town refuses to remove their judgment. Hawthorne remarks that ``in all [Hester's] intercourse with society\ldots Every gesture, every word, and even the silence of those with whom she came into contact, implied, and often expressed that she was banished\ldots'' (81). All of this good that Hester Prynne performs is lost to the larger view of society because they have so thoroughly stigmatized her as a sinner that any good she does is already tainted by the Hell, and everyone conveyed to her the truth that ``she was banished,'' both from the social circles of Boston, but also from the greater goals of society. By attempting to purge sin from the inhabitants of Boston, the Puritans have instead removed the weight of the good that Hester Prynne did, and prevented any stigmata from being removed from her charity work. Like the utopic projects of the \nth{19} century, Puritan Boston, in Hawthorne's view, has transitioned from a 'perfect society' into a society that ostracizes those who do not absolutely conform to the idealistic view of those who are in charge. In Boston's case, this is the 'living saints' while in the \nth{19} century, these are the founders and zealots. A utopia based on ideas that contradict or stigmatize basic human good inevitably decays into a rule by the 'better,' who often attempt to crush the human spirit of those they lord over.\\
Once Hester Prynne receives the scarlet letter, she becomes supernaturally aware of the sins of others, such that the sign on her chest burns. As Hester walks through the streets of the town, sometimes ``she felt an eye - a human eye - upon the ignominious brand, that seemed to give a momentary relief, as if half her agony were shared. The next instant, back it tall rushed again, with still a deeper throb of pain; for, in that brief interval, she had sinned anew. Had Hester sinned alone?'' (83). When her fellow sinners glance at her letter, Hester becomes aware of their travesty in a moment of shared suffering. Because they have kept their sin secret, Hester is relieved for a moment by their lies taking on some of her suffering, only to have it crash back down as ``she had sinned anew.'' She recognizes the hypocrisy endemic to a society in which everyone sins but also attempt to convince the outside world that they are pure and deserving, and in a way this makes her suffering more desolate because she is surrounded by those who could share her burden, but is unable to connect with them due to their lies. Perhaps the biggest socialist movement during the \nth{19}, the Fourier movement, had several hundred utopian communities spread throughout the United States, and was for a period during the mid-\nth{19} century a massive popular movement as well. Hawthorne himself was a member of the Brook Farm socialist community~\parencite{preucel}. The movement was based on the ideas of French intellectual Charles Fourier, whose social discourse was all the rage in the formative years of the US~\parencite{nako}. The majority of these na{\"i}ve fledgeling societies would fail in the first five years due to a number of reasons that Hawthorne alludes to in \textit{The Scarlet Letter}.
Additionally, Hawthorne uses vivid imagery to convey the impact that utopia has on the natural way of the human soul.\\
\center{WORK IN PROGRESS}
\end{mla}
\end{document}
