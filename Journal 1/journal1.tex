\documentclass[11pt, letterpaper]{article}

\usepackage{ifpdf}
\usepackage{mla}

\begin{document}
\begin{mla}{Lee}{Mracek}{Mary Hill}{American Studies}{\today}{Journal from 1/15/15 on \textit{The Jungle}}
    The author argues that de-regulation of the meat industry will only cause more problems domestically, and pushes for increased budget for the FDA and more regulatory laws. He uses the horrible standards of the past, evidenced in Upton Sinclair's \textit{The Jungle} to describe the industry when no regulation existed. By contrasting the harsh regulation created after \textit{The Jungle} with the lesser regulation today, the author projects a descent back into maggoty meat and poisoned rats. Without regulation, he argues, corporations in the meat industry naturally decay into the intolerable standards of the past and again doom the American people to eating dead rats and human meat.

    Preventing deregulation of key industries is absolutely crucial in order to maintain the health and standards of our nation. The profit motive in meat packing tends towards minimizing safety; if you use poisoned rats, that's less meat you have to trash. The author's analysis of the general trends in safety and sanitation help to corroborate this understanding by showing a link between decreasing regulation and declining standards. By hitting the American people ``in their stomachs,'' as Sinclair did many years before, the author of this article attempts to convince the American people that they are indeed once again in peril from the corporate need to make more and more money unless enough regulation is put into place. The author fears that the American people have forgotten the roots of the FDA and health regulation, and must be reminded.

    In \textit{The Jungle}, as well as our class studies on American industrialism and trusts, it is clear that a perfectly lassez-faire market leads to monopolies and low standards in a desperate drive to increase profit margin. Rockefeller taking off a dot of solder is driven by the same motivations as using poisoned rats to make food, indicating that the trend is towards negligence for profits. As the author tries to show us, the profit driven world of the early 1900s still exists in our modern world, only hidden behind the regulation that exists in part because of stories like \textit{The Jungle}. \textit{The Jungle} brought important information to the American people regarding the state of their food and their environment which we perhaps have lost some of today.
\end{mla}
\end{document}

