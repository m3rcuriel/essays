\documentclass[11pt,letterpaper]{article}

\usepackage{ifpdf}
\usepackage{mla}
\usepackage[utf8]{inputenc}

\begin{document}
\begin{mla}{Lee}{Mracek}{Mary Hill}{American Studies}{\today}{Stats and \textit{The Things They Carried}}
    Hans Rosling and his non-profit, gapminder.com, have committed themselves to reducing the rampant statistical misconceptions that the majority of people in ``first world countries'' have about the world. For instance, in samples of the Swedish population, the U.S population, and random studies of the crowd at TED, the people were far less effective at determining the correct answer to several questions such as ``Has the number of people living in poverty a) doubled b) stayed the same or c) almost halved since 1960.'' The chimpanzees, of course, put 33\% on each answer, while the Swedes put the about 40\% on each b and c, and TED stayed mostly with the Swedes. Interestingly, at a media conference in 2013, the media almost outperformed the chimpanzees. Hans attempts to show people that the world has changed, and their information and assumptions are outdated and incorrect.

    In a way, \textit{The Things They Carried} also tells the story of ignorance and superstition. Much like the viewers of Hans' TED talk do not understand the realities of the modern era, Timmy and his company do not really understand the realities of their era, and are fighting for a cause they are not really sure that they understand. Tim O'Brian speaks about the weird surreal feeling of war, and the odd blurring of everything that Alpha Company experience in Vietnam, both of which in the book lead to confusion and a weird expression of emotion and pain that people do not really have during peacetime. The misinformation that we face in the modern era is the same as the misconceptions in the Vietnam War era that led to such a brutal conflict and ``World Police'' view in the first place.

    Tim O'Brian and Hans Rosling each fight different aspects of the same war of information. For O'Brian, the key is attacking information at it's emotional source - in the thought of the American people regarding war and Vietnam in particular, and the telling of war stories. Rosling, on the other hand, attacks these misconceptions at the historical or geopolitical level using statistics to disprove the beliefs of his listeners. They fight the same battle, and Rosling's insistence that data doesn't permeate into the minds of the public is similar to O'Brian's idea regarding the public's ignorance of war stories. The public, in both of their eyes, as fundamental misconceptions that need to be solved in order to improve as a society in the modern age.

    O'Brian and Rosling's respective works parallel each other at a fundamental, motivational level. While O'Brian struggles to reduce ignorance in a world rapidly, it seems, losing it's collective sense of empathy, Rosling struggles to reduce ignorance in a world that seems to be losing grasp on reality. O'Brian tells us that the emotional impact of war is significant and one that cannot be measured simply by the physical things that each soldier carries, while Rosling tries to tell us that our understanding of the world is wrong and outdated through the misconceptions and biases that we are given from childhood onwards.
\end{mla}
\end{document}

