\documentclass[letterpaper,11pt]{article}

\usepackage{ifpdf}
\usepackage{mla}
\usepackage[T1]{fontenc}
\usepackage[utf8]{inputenc}

\begin{document}
\begin{mla}{Lee}{Mracek}{Mary Hill}{American Studies}{\today}{On Rational Decisions: Journal from 5/11/15}
As humans, life seems like it should be wholly rational. Science is perfectly rational, and the laws of nature at least seem rational or governed by chance. No one, however, can make a perfectly rational decision. Imperceptible biases, lack of information, and general stress or tension can make arriving at a rational decision very challenging. I think, also, that the human mind in itself cannot make perfectly rational decisions because we and society define our own weights, which any ``rational'' decisions must be based on. But how do we know those weights are not rational? Simple: They themselves are based on calculations off of other weighting systems which themselves are not rational. One might make a decision based on personal value placed on loyalty or integrity. How is that value determined? By more values, of course! You can follow this down to impulses in nerves that originate in the strength of neural connections that in themselves are more or less random simply by the shear number of possible variables that can influence them.

Our brain is a very complex neural network-like system, and unlike more boring decision boring decision tree systems (which is rational assuming its nodes are rational), neural networks are trained to see patterns. You do not need to understand a system to understand that x -> y or an orange should be under \$10. Pricing makes sense to us intuitively without understanding the complex economic or social factors which determined that pricing. A neural network, though, does not guarantee ``correct'' or ``rational'' answers; it just approaches it. Recently, researchers trained a neural network to perfectly park a trailer in a loading dock using a truck that can only turn and back up. The neural network succeeded, but only after being trained by starting randomly and failing until it could recognize a successful output pattern for any given input pattern. That's what the brain does, but with an incredibly powerful feedforward system to take the random out of the initial guess. Not only can we output patterns (decisions) based on similar input patterns, but can also make complete guesses based on feedfoward responses from our prior experiences.

As such, we can make good decisions, but they are not rational. Rational decisions cannot be performed by a system such as the human brain because the human brain relies on this pseudo-guess to make any decision. All of your decisions are just guesses, in the end, to at least some degree because that was the best evolutionary option. A big \textit{thing} is running towards you, you need to run before you can figure out what it is. That, of course, is a very simple case, but it applies to larger situations as well. We can make decisions based on a big picture view because we can process that sort of input rather than simply make wholly rational decisions on small minute pieces of input.

Evolutionarily, the human brain needs to be able to react to unknown stimulus. Unknown stimulus, after all, is what develops us from baby to adult.  

The brain lies when it tries to be rational. The brain cannot be rational. It cannot possibly take all the input from the environment and arrive at some ideal situation based on absolutes. Life does not work that way. Fortunately, what we have is probably better in the end based on our evolutionary history and ability to adapt to a world as crazy and quick-changing as our own. Emotions cloud decisions, stress clouds decisions, and at the bottom of it our brain simply is not wired to make perfect decisions -- it's wired to keep us alive.
\end{mla}
\end{document}
